\newcommand{\institut}{}
\newcommand{\fachgebiet}{Regelungstechnik}
\newcommand{\veranstaltung}{Praktikum Grundlagen der Regelungstechnik}
\newcommand{\pdfautor}{Dirk Barbendererde (321 836), Boris Henckell (325 779)}
\newcommand{\autor}{Dirk Barbendererde (321 836)\\ Boris Henckell (325 779)}
\newcommand{\pdftitle}{Praktikum Regelungstechnik  Versuch 1a}
\newcommand{\prototitle}{Praktikum Regelungstechnik \\ Versuch 1a}
\newcommand{\aufgabe}{}

\newcommand{\gruppe}{Gruppe: G1 Di 12-14}
\newcommand{\betreuer}{Betreuer: Christian Schmuck}



\input{../../packages/tu_header_8}


% \lstlistoflistings
\definecolor{darkgray}{rgb}{0.95,0.95,0.95}
\lstset{language=Scilab}
\lstset{inputencoding=utf8}
%\lstset{extendedchars=true} % Umlaute an der richtigen stelle und nicht am Anfang ausgeben
\lstset{backgroundcolor=\color{darkgray}}
\lstset{numbers=left, numberstyle=\tiny, stepnumber=1, numbersep=7pt, breaklines=true}
\lstset{keywordstyle=\color{red}\bfseries\emph}
\lstset{
breaklines,
numbers=left,
frame=single,
xleftmargin=-2cm,
xrightmargin=-1.5cm
}
% enables UTF-8 in source code: (dirty, dirty hack)
\lstset{literate=
    %Deutsch
    {ä}{{\"a}}1 {ö}{{\"o}}1 {ü}{{\"u}}1 {Ä}{{\"A}}1 {Ö}
    {{\"O}}1 {Ü}{{\"U}}1 {ß}{\ss}1
    %Türkisch
    {â}{{\^{a}}}1 {Â}{{\^{A}}}1 {ç}{{\c{c}}}1 {Ç}{{\c{C}}}1 {ğ}{{\u{g}}}1 {Ğ}{{\u{G}}}1 {ı}{{\i}}1 {İ}{{\.{I}}}1 {ö}{{\"o}}1 {Ö}{{\"O}}1 {ş}{{\c{s}}}1
    {Ş}{{\c{S}}}1 {ü}{{\"u}}1 {Ü}{{\"U}}1
    %Polish
    {ą}{{\k{a}}}1 {ć}{{\'c}}1 {ę}{{\k{e}}}1 {ł}{{\l{}}}1 {ń}{{\'n}}1 {ó}{{\'o}}1 {ś}{{\'s}}1 {ż}{{\.z}}1 {ź}{{\'z}}1 {Ą}{{\k{A}}}1 {Ć}{{\'C}}1
    {Ę}{{\k{E}}}1 {Ł}{{\L{}}}1 {Ń}{{\'N}}1 {Ó}{{\'O}}1 {Ś}{{\'S}}1 {Ż}{{\.Z}}1 {Ź}{{\'Z}}1
    %Spanish
    {á}{{\'a}}1 {é}{{\'e}}1 {í}{{\'i}}1 {ó}{{\'o}}1 {ú}{{\'u}}1 {ñ}{{\~n}}1
}

%     \lstinputlisting{./praktikum6.sce}



%---------------------------------------------------------------------
%---------------------------------------------------------------------
%---------------------------------------------------------------------

\section{Vorbereitungsaufgaben}
\begin{quote}
	\subsection{lineare Zustandsmodell}
	\begin{quote}
		Aufgabe:\\
		Leiten Sie das vollständige lineare Zustandsmodell für die Regelstrecke her. Stellen Sie hierfür zunächst
		die Differentialgleichung für den Ankerstrom und die Bilanzgleichung der Drehmomente an der Massenscheibe auf. Welche
		Komponenten hat der Zustandsvektor?\\
		
		
        
	\end{quote}
	
	\subsection{Blockschaltbild}
	\begin{quote}
		Aufgabe:\\
		Zeichnen Sie das Blockschaltbild der aus Leistungsverstärker und Gleichstrommaschine mit Schwungscheibe
		bestehenden Regelstrecke und beschriften Sie die Signalpfeile mit den zugehörigen Größen. Es sollen ausschließlich Summations-,
		Verstärkungs- und Integratorblöcke verwendet werden.\\
		
		
		
	\end{quote}
	
	\subsection{Zustandsraummodell}
	\begin{quote}
		Aufgabe:\\
		Der Leistungsverstärker soll nun durch die statische Verstärkung $V$ approximiert werden, da er eine
		schnelle Dynamik besitzt, die vernachlässigt werden kann. Wie lautet das resultierende Zustandsraummodell? Verwenden
		Sie dieses Modell für alle weiteren Betrachtungen.\\
		
		
		
	\end{quote}
	
	\subsection{Übertragungsfunktion}
	\begin{quote}
		Aufgabe:\\
		Bestimmen Sie die Übertragungsfunktion $\overline{G}_{ui}(s) = \frac{I_A(s)}{U(s)}$ zwischen der
		Eingangsspannung $u$ des Leistungsverstärkers und dem Ankerstrom $i_A$. Vereinfachen Sie die 
		Übertragungsfunktion $\overline{G}_{ui}$ durch Vernachlässigung sehr schneller Dynamikanteile der Strecke, d.h.
		Vernachlässigung von Polen mit betragsmäßig sehr hohem negativen Realteil. Wie lautet die resultierende
		vereinfachte Übertragungsfunktion $G_{ui}$? Warum ist diese Vereinfachung zulässig? Verwenden Sie im Folgenden
		die vereinfachte Übertragungsfunktion $G_{ui}$.\\
				
        
	\end{quote}
	
	\subsection{Wurzelortskurve}
	\begin{quote}
		Aufgabe:\\
		Wiederholen Sie das Thema "`Wurzelortskurve"'!\\
		

    \end{quote}
    
    \subsection{Reglerentwurf}
    Entwerfen Sie einen mit einem PT1-Glied verketteten PI-Regler $K_i(s) = k_i \frac{s-s_{0,i}}{s} \frac{-s_1}{s-s_1}$
    \begin{quote}
        \subsubsection{Vorteil dieses Reglers}
            
        \begin{quote}
            Aufgabe:\\
            Welchen Vorteil besitzt die gewählte Reglerstruktur gegenüber einem reinen PI-
            Regler? Welche Effekte erwarten Sie für einen reinen PI-Regler im geschlossenen
            Regelkreis?\\
            
        \end{quote}


        \subsubsection{Polstelle $s_1$}
        \begin{quote}
            Aufgabe:\\
            Platzieren Sie die Polstelle $s_1$ des Tiefpasses anhand von Überlegungen mit der
            Wurzelortskurve. Ihr Ziel sollte eine möglichst schnelle Reaktion des Regelkreises
            sein.\\
            
        \end{quote}
        
        
        \subsubsection{Wurzelortskurve}

        \begin{quote}
            Aufgabe:\\
            Legen Sie nun weiterhin anhand der Wurzelortskurve die Nullstelle $s_{0,i}$
            fest, sodass bei einer geeigneten Wahl von $k_i$ des Reglers
            das Potential besteht, einen schnellen Regler zu entwerfen.\\
        
        \end{quote}
            
        
        \subsubsection{Strungantwort}

        \begin{quote}
            Aufgabe:\\
            Die Sprungantwort des geschlossenen Kreises soll nach $0,02s$ nur noch eine Regelabweichung von $5\%$
            aufweisen. Bestimmen Sie $k_i$ mit Hilfe von Simulationen der Führungssprungantwort.\\
        
	 	
        \end{quote}

    \end{quote}
    
    
    \subsection{Störübertragungsfunktion}
    \begin{quote}
        Aufgabe:\\
        Berechnen Sie die Störübertragungsfunktion $\frac{I_A(s)}{D_u(s)}$ des Regelkreises, die den Einfluss
        einer Störspannung $d_u$ am Eingang des Leistungsverstärkers auf den Ankerstrom $i_A$ beschreibt.
        Berechnen Sie die Störübertragungsfunktion. Simulieren Sie die Störsprungantwort.\\
        
	 	
    \end{quote}    
    
    
    
    \subsection{Sensitivitäts- und komplimentäre Sensitivitätsfunktion}
    \begin{quote}
        Aufgabe:\\
        Zeichnen Sie die Amplitudenfrequenzgänge der Sensitivitätsfunktion $S_i(s)$ und der komplementären
        Sensitivitätsfunktion $T_i (s)$ des geschlossenen Regelkreises in dem Bereich $\omega = 10^{-3}\frac{rad}{s}
        \ldots 10^{3}\frac{rad}{s}$ um das Verhalten der Stromregelung für den gesamten Frequenzbereich beurteilen zu
        können.\\
        Machen Sie Aussagen darüber, für welche Frequenzbereiche der Referenz- und Störgröße gutes Regelverhalten
        erzielt wird und in welchen Frequenzbereichen auftretendes Messrauschen sich kaum auf die Regelgröße auswirkt.\\
        
                        
	\end{quote}
		
	\subsection{Blockstaltbild des Reglers}
    \begin{quote}
        Aufgabe:\\
        Erstellen Sie ein Blockschaltbild des Reglers auf Basis zweier Integratoren, in welchem der PI-Anteil und PT1-Anteil in Reihe geschaltet
        sind. Die Parameter $k_i$, $s_{0,i}$, $s_1$ sollen direkt in die Verstärkungsblöcke eingehen. Implementieren Sie
        den kompletten Regelkreis als Scicos-Diagramm und bringen Sie Ihre Simulationsdateien zum Durchführungtermin
        mit!
        
	 	
    \end{quote}
    
    \subsubsection{Regelparameter überprüfen}
        Aufgabe:\\
        Bei Sprüngen der Führungsgröße $r_\omega$ von $0$ auf Werte bis zu $180 \frac{rad}{s}$ soll die
        Stellgrößenbeschränkung von $−5V < u < 5V$ nicht verletzt werden. Überprüfen Sie simulativ, ob Ihr Regler
        diese Forderung erfüllt. Falls nicht, korrigieren Sie ihre Reglerparameter, sodass die Forderung erfüllt wird.
        Wie lautet die Übertragungsfunktion $\overline{T}_\omega$ des bis hierhin entworfenen geschlossenen
        Regelkreises?
        \begin{quote}
            
        \end{quote}
    
    


        
    
  
\end{quote}

%--------------------------------------------------------------------
%--------------------------------------------------------------------

\section{Versuch}
\begin{quote}
    
\end{quote}

%--------------------------------------------------------------------
%--------------------------------------------------------------------

\section{Ergebnisse}
\begin{quote}
    
\end{quote}

%--------------------------------------------------------------------
%--------------------------------------------------------------------



% \begin{quote}
%     \lstinputlisting[
%         caption={Scilab-script},
%         label=lst:scilab]
%         {./Scilab/Motor.sce}
%         
% \end{quote}

%--------------------------------------------------------------------
%--------------------------------------------------------------------



\end{document}
