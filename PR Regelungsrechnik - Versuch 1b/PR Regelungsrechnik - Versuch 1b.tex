\newcommand{\institut}{}
\newcommand{\fachgebiet}{Regelungstechnik}
\newcommand{\veranstaltung}{Praktikum Grundlagen der Regelungstechnik}
\newcommand{\pdfautor}{Dirk Barbendererde (321 836), Boris Henckell (325 779)}
\newcommand{\autor}{Dirk Barbendererde (321 836)\\ Boris Henckell (325 779)}
\newcommand{\pdftitle}{Praktikum Regelungstechnik  Versuch 1b}
\newcommand{\prototitle}{Praktikum Regelungstechnik \\ Versuch 1b}
\newcommand{\aufgabe}{}

\newcommand{\gruppe}{Gruppe: G1 Di 12-14}
\newcommand{\betreuer}{Betreuer: Christian Schmuck}



\input{../../packages/tu_header_8}


% \lstlistoflistings
\definecolor{darkgray}{rgb}{0.95,0.95,0.95}
\lstset{language=Scilab}
\lstset{inputencoding=utf8}
%\lstset{extendedchars=true} % Umlaute an der richtigen stelle und nicht am Anfang ausgeben
\lstset{backgroundcolor=\color{darkgray}}
\lstset{numbers=left, numberstyle=\tiny, stepnumber=1, numbersep=7pt, breaklines=true}
\lstset{keywordstyle=\color{red}\bfseries\emph}
\lstset{
breaklines,
numbers=left,
frame=single,
xleftmargin=-2cm,
xrightmargin=-1.5cm
}
% enables UTF-8 in source code: (dirty, dirty hack)
\lstset{literate=
    %Deutsch
    {ä}{{\"a}}1 {ö}{{\"o}}1 {ü}{{\"u}}1 {Ä}{{\"A}}1 {Ö}
    {{\"O}}1 {Ü}{{\"U}}1 {ß}{\ss}1
    %Türkisch
    {â}{{\^{a}}}1 {Â}{{\^{A}}}1 {ç}{{\c{c}}}1 {Ç}{{\c{C}}}1 {ğ}{{\u{g}}}1 {Ğ}{{\u{G}}}1 {ı}{{\i}}1 {İ}{{\.{I}}}1 {ö}{{\"o}}1 {Ö}{{\"O}}1 {ş}{{\c{s}}}1
    {Ş}{{\c{S}}}1 {ü}{{\"u}}1 {Ü}{{\"U}}1
    %Polish
    {ą}{{\k{a}}}1 {ć}{{\'c}}1 {ę}{{\k{e}}}1 {ł}{{\l{}}}1 {ń}{{\'n}}1 {ó}{{\'o}}1 {ś}{{\'s}}1 {ż}{{\.z}}1 {ź}{{\'z}}1 {Ą}{{\k{A}}}1 {Ć}{{\'C}}1
    {Ę}{{\k{E}}}1 {Ł}{{\L{}}}1 {Ń}{{\'N}}1 {Ó}{{\'O}}1 {Ś}{{\'S}}1 {Ż}{{\.Z}}1 {Ź}{{\'Z}}1
    %Spanish
    {á}{{\'a}}1 {é}{{\'e}}1 {í}{{\'i}}1 {ó}{{\'o}}1 {ú}{{\'u}}1 {ñ}{{\~n}}1
}

%     \lstinputlisting{./praktikum6.sce}



%---------------------------------------------------------------------
%---------------------------------------------------------------------
%---------------------------------------------------------------------

\section{Vorbereitungsaufgaben}
\begin{quote}
	\subsection{Übertragungsfunktion}
    Aufgabe:\\
    Ermittlen Sie die Übertragungsfunktion $G_\omega^{'} (s) = \frac{\Omega(s)}{R_i (s)}$ der nun relevanten
    Regelstrecke, die den unterlagerten Stromregelkreis aus Versuchsteil 1a enthält.\\
	\begin{quote}
		
		
        
	\end{quote}
	
	\subsection{Regler}
    Aufgabe:\\
    Als Drehzahlregler soll ein PI-Glied\\
    \begin{equation*}
        \begin{split}
            K_\omega (s) = k_\omega \frac{s-s_{0,\omega}}{s}, \hspace{2em} k_\omega, s_{0,\omega} \in \mathbb{R}
        \end{split}
    \end{equation*}
    verwendet werden.
	\begin{quote}
		
		\subsubsection{Begründen}
        Aufgabe:\\
        Begründen Sie, warum diese Reglerstruktur sinnvoll ist.\\
		\begin{quote}
			
		\end{quote}
		
		\subsubsection{Bode-Diagramm}
        Aufgabe:\\
        Der Reglerentwurf soll nach dem Frequenzkennlinienverfahren durchgeführt werden. Zeichnen Sie dafür zunächst
        das Bode-Diagramm von $G_\omega^{'}$.\\   
        \begin{quote}
                        
        \end{quote}

        \subsubsection{Reglerentwurf}
        Aufgabe:\\        
        Wählen Sie zunächst das aus der Vorlesung bekannte Entwurfsvorgehen, indem Sie mit der Nullstelle $s_{0,\omega}$
        des Reglers die langsamste Polstelle der Strecke kürzen. Bestimmen Sie dann mit Hilfe von Simulationen die
        Verstärkung $k_\omega$ so, dass der geschlossene Regelkreis eine Ausregelzeit von etwa $0.6$ Sekunden aufweist
        und das Ü̈berschwingen einen Wert von $20\%$ nicht übersteigt. Rufen Sie sich in Erinnerung, wie die
        ``Kenngrößen'', ``Überschwingweite'' , ``Ausregelzeit'' , ``Phasenreserve'' und ``Durchtrittsfrequenz''
        miteinander in Beziehung stehen!
		\begin{quote}
	
		\end{quote}
        
        \subsubsection{Erstellen der Führungsübertragungsfunktion}
        Aufgabe:\\
        Bei Sprüngen der Führungsgröße $r_\omega$ von $0$ auf Werte bis zu $180 \frac{rad}{s}$ soll die
        Stellgrößenbeschränkung von $−5V < u < 5V$ nicht verletzt werden. Überprüfen Sie simulativ, ob Ihr Regler
        diese Forderung erfüllt. Falls nicht, korrigieren Sie ihre Reglerparameter, sodass die Forderung erfüllt wird.
        Wie lautet die Übertragungsfunktion $\overline{T}_\omega$ des bis hierhin entworfenen geschlossenen
        Regelkreises?
        \begin{quote}
            
        \end{quote}
        
        \subsubsection{Erstellen der Störübertragungsfunktion}
        Aufgabe:\\
        Berechnen Sie die Störübertragungsfunktion $\overline{G}_{m\omega} (s) = \frac{\Omega(s)}{M_L (s)}$ vom
        Lastmoment $m_L$ auf die Winkelgeschwindigkeit $\omega$. Simulieren Sie die Störsprungantwort. Warum ist das
        Störverhalten des entworfenen Regelkreises nicht brauchbar? Was fällt Ihnen auf, wenn Sie die Pole von
        $\overline{T}_\omega$ und $\overline{G}_\omega$ vergleichen?
		\begin{quote}
			
		\end{quote}
		
		\subsubsection{Korrektur der Führungs- und Störübertragungsfunktion}
		Aufgabe:\\
	    Überlegen Sie mit Hilfe der Wurzelortskurve, wie die Nullstelle $s_{0,ω}$ des Reglers verschoben werden muss, um
	    das Störverhalten zu verbessern. Wählen Sie $s_{0,ω}$ und $k_\omega$ neu, sodass alle vorher genannten
	    Forderungen an das Führungsverhalten weiterhin erfüllt werden. Wie lauten die neue
	    Führungsübertragungsfunktion $T_\omega$ und die neue Störübertragungsfunktion $G_{m\omega}$ und deren Pole?
		\begin{quote}
			
		\end{quote}
		
	\end{quote}
	
	\subsection{Anti-Windup-Schaltung in Scicos erstellen}
	Aufgabe:\\
    Erweitern Sie Ihre Reglerstruktur in Scicos um eine Anti-Windup-Schaltung nach Abbildung 11. Beziehen Sie sowohl
    den äußeren als auch den inneren Regler in die Schaltung ein. Realisieren Sie dafür den inneren und äußeren
    PI-Regler in einer Parallelform mit P- und I-Anteil.
	\begin{quote}
		
		
	\end{quote}
	
\end{quote}

%--------------------------------------------------------------------
%--------------------------------------------------------------------

\section{Versuch}
\begin{quote}
    
\end{quote}

%--------------------------------------------------------------------
%--------------------------------------------------------------------

\section{Ergebnisse}
\begin{quote}
    
\end{quote}

%--------------------------------------------------------------------
%--------------------------------------------------------------------



% \begin{quote}
%     \lstinputlisting[
%         caption={Scilab-script},
%         label=lst:scilab]
%         {./Scilab/Motor.sce}
%         
% \end{quote}

%--------------------------------------------------------------------
%--------------------------------------------------------------------



\end{document}
